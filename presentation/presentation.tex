\documentclass{beamer}

\usepackage{hyperref}
\usepackage{listings}
\usepackage{color}

% Colours for beamer.
\setbeamercolor{frametitle}{fg=orange}
\setbeamertemplate{itemize item}{\color{orange}$\blacksquare$}
\setbeamertemplate{itemize subitem}{\color{orange}$\blacktriangleright$}

% Colours for syntax highlighting
\definecolor{syntax_red}{rgb}{0.7, 0.0, 0.0} % For strings
\definecolor{syntax_green}{rgb}{0.15, 0.5, 0.25} % For comments
\definecolor{syntax_purple}{rgb}{0.6, 0.0, 0.45} % For keywords


% Haskell settings for lstlisting
\lstset{language=Haskell,
basicstyle=\ttfamily,
keywordstyle=\color{syntax_purple}\bfseries,
stringstyle=\color{syntax_red},
commentstyle=\color{syntax_green},
numbers=none,
numberstyle=\color{black},
stepnumber=1,
numbersep=10pt,
tabsize=4,
showspaces=false,
showstringspaces=false}

\author{
  Beck, Calvin\\
  \href{mailto:hobbes@ualberta.ca}{hobbes@ualberta.ca}
}

\begin{document}

\begin{frame}
  \frametitle{A Journey Through Types}
  \maketitle
\end{frame}

\section{Introduction}

\begin{frame}
  \frametitle{What is this Talk about?}

  Types! This presentation hopes to address the following:

  \begin{itemize}
  \item How types make things easier to write.
  \item How types can help you write correct software.
  \end{itemize}

  \pause

  Somewhat of a whirlwind introduction. Let me know if you're lost,
  because this talk is all over the place!
\end{frame}

\section{What are types, and how do they help us?}

\begin{frame}[fragile]
  \frametitle{What are we trying to solve?}

  You think that this is normal...

  \pause

  \begin{lstlisting}[frame=single, language=Python, breaklines=true]
    Traceback (most recent call last):
      File "<stdin>", line 1, in <module>
    TypeError: 'NoneType' object is not subscriptable
  \end{lstlisting}

  \pause

  \huge{... It's not!}
\end{frame}

\begin{frame}
  \frametitle{What is a type?}

  A type describes what a value ``is''.

  \pause

  You have probably heard of this as ``how the bits are stored in memory.''

  \pause

  It's a bit more than that!

  \pause

  \begin{itemize}
  \item Tell us how to use values.
    \pause

    \begin{itemize}
    \item Tells us what operations are defined on the types.
    \item Can you add things of this type?
    \item Can a function take a value of this type as an argument?
    \item What kind of stuff does this function return?
    \end{itemize}

  \pause

  \item Documentation

  \pause

  \item Rejection of general nonsense: \(357^\text{circles}\)

  \pause

  \begin{itemize}
  \item \huge{NO MORE NULL REFERENCE EXCEPTIONS!}
  \end{itemize}
  \end{itemize}
\end{frame}

\section{The types you may have seen}

\begin{frame}[fragile]
  \begin{lstlisting}[frame=single, language=Python, breaklines=true, basicstyle=\ttfamily\tiny]
     def my_sort(xs):
         if xs == []:
             return xs
         else:
             first_elem = xs[0]
             rest = xs[1:]

             smaller = my_sort([x for x in rest if x <= first_elem])
             larger = my_sort([x for x in rest if x > first_elem])

             return smaller + [first_elem] + larger


     def my_factorial(n):
         if n == 0:
             return 1
         else:
             return n * my_factorial(n-1)
  \end{lstlisting}

  \begin{itemize}
  \item No types to help document functions.
  \item No types to catch errors at runtime.
    \begin{itemize}
    \item Tests can help...
    \item But it's nice to not have to worry about certain errors at all.
    \end{itemize}
  \end{itemize}
\end{frame}
\end{document}
